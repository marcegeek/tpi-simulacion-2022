\documentclass{article}
\usepackage[utf8]{inputenc} % allow utf-8 input
\usepackage[T1]{fontenc}    % use 8-bit T1 fonts
\usepackage[spanish]{babel} % idioma español
\usepackage{hyperref}       % hyperlinks
\usepackage{url}            % simple URL typesetting
\usepackage{booktabs}       % professional-quality tables
\usepackage{amsmath}        % useful for serious math
\usepackage{amsfonts}       % blackboard math symbols
\usepackage{nicefrac}       % compact symbols for 1/2, etc.
\usepackage{microtype}      % microtypography
\usepackage{kpfonts}        % use the same fonts for text and math
\usepackage{graphicx}
\graphicspath{ {./images/} {./images/generated/} }
\usepackage{float}          % para fijar figuras con \begin{figure}[H] ... \end{figure}
\usepackage{subcaption}     % para poder crear subfiguras con \begin{subfigure}{ancho}...\end{subfigure}
\usepackage{import}         % para facilitar la inclusión de los demas archivos
\usepackage[
  backend=biber,
  %style=numeric,
  %sorting=none,
  style=ieee,
  citestyle=numeric-comp,
]{biblatex}                 % bibliografía con BibLaTex (soporte de idiomas con babel y otras mejoras)
\bibliography{referencias}  % buscar bibliografía en referencias.bib
